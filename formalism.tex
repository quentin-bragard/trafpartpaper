\section{FORMALIZATION OF THE PROBLEM STATEMENT}
\label{sec:form}

In this paper we conduct a comparison of different methods of partitioning a road network graph based on some metrics. This section defines what a road network graph is and presents the formal definition of the metrics.

\subsection{The Road Network Graph}
\label{sec:form-road-netw-grap}
The road network of a city can be represented by a directed cyclic graph~\cite{holden1995mathematical} $G(V, E)$ where $V$ denotes the vertex set and $E$ denotes the edge set. Every edge $e_{ij} \in E$ in the graph represents a unidirectional road in the city that connects intersection $v_i$ to intersection $v_j$. Every vertex $v_i \in V$ denotes an intersection of two or more roads. A weight $w_{ij}$ is associated with the edge $e_{ij}$ that is representative of the traffic that flows through that road. As discussed in Section~\ref{sec:futu}, we intend to increase the number of weights that can be associated with every edge to be able to represent the number of lanes, length of the road, importance of the road etc. as part of our future work.

\subsection{Partitioning a graph}
\label{sec:form-part}
For the sake of completeness of the graph definition, we also assign weights to the vertices denoted by $W_i$, which is defined as follows :
\begin{equation}
\label{eq:vertex-weight}
W_i = \sum\limits_{\forall j \in neighbours(i)} w_{ij}
\end{equation}

\noindent where $neighbours(i)$ is the set of all nodes in $V$ that receive an outgoing edge from $v_i$. 

The problem of partitioning a graph has a 2 dimensional space, where one axis represents the node ID(representing the intersection) and the other represents the partition ID. Each point in this 2-D space represents an \textit{\{intersection,partition\}} pair which implies this intersection is mapped onto this partition. We define a ``state'' to be a collection of $|V|$ points such that each intersection is mapped to exactly one partition. The total number of possible ``states'' in this discrete space is $|V|^{p}$, where $p$ is the number of required partitions.

\subsection{Formalizing the objective function}
\label{sec:form-obje-func}

To effectively perform a distributed road network simulation, one must fully understand the effects of dividing the workload(the road network graph to be simulated) into the required number of partitions. To this end, we define three heuristics\{cite papers\} that are widely accepted in the community, to measure the effectiveness of a ``partitioning scheme''. 

We define a partitioning scheme on the given road network $G$, $\lambda_G \in \Lambda_G$(where $\Lambda_G$ is the set of all possible partitioning schemes of the given graph $G$ onto $p$ partitions) as a function that maps every node $v_i$ from the road network graph to a partition. Please note that we will use the terms ``node'', ``vertex'' and ``intersection'' interchangeably.
\begin{equation}
\label{eq:part-sche}
\lambda_G(v_i) \in \{P_0,\dots,P_{p-1}\}, \forall v_i \in V
\end{equation}

\noindent where $P_i$ denotes the $i^{th}$ partition. For the following discussion we denote the weight of the $i^{th}$ partition($P_i$) as $w(P_i)$. One of the primary objectives of road network partitioning for distributed simulation is to minimize the execution time of the simulation time. To this end, we aim to minimize the weight of the \textbf{heaviest partition($\lambda_G^{max}$)} as described by the following equation,

\begin{equation}
\label{eq:metric1}
\lambda_G^{max} = max(w(P_i)), \forall P_i \in \{P_0,\dots,P_{p-1}\}
\end{equation}

\noindent The execution time of a distributed simulation also depends on the amount of \textbf{inter-partition communication}($\lambda_G^{comm}$). In order to reduce this inter-partition communication, we minimize the following metric,

\begin{equation}
\label{eq:metric2}
\lambda_G^{comm} = \sum\limits_{i \in \{v_0,\dots,v_{|V|-1}\}}{\sum\limits_{j \in \{v_0,\dots,v_{|V|-1}\}}{w(e_{ij})}} , s.t. \lambda_G(i) \neq \lambda_G(j)
\end{equation}

\noindent The final metric that we consider is the \textbf{evenness}($\lambda_G^\sigma$) of the partitions. In a scenario in which the underlying architecture consists of homogeneous execution units(i.e. a vertex of the road network graph takes equally long to simulate on any machine in the underlying architecture) and where communication plays a less significant role(i.e. powerful communication backbone), this metric plays a very important role as it defines how varied each machine is loaded.

\begin{equation}
\label{eq:metric3-mean}
\lambda_G^\mu = \frac{\sum\limits_{P_i \in \{P_0,\dots,P_{p-1}\}} w(P_i)}{p}
\end{equation}

\begin{equation}
\label{eq:metric3}
\lambda_G^\sigma = \frac{\sqrt{\sum\limits_{P_i \in \{P_0,\dots,P_{p-1}\}} (w(P_i) - \lambda_G^\mu)^2}}{p}
\end{equation}

\noindent The importance of these metrics vary hugely depending on the underlying architecture onto which these partitions are mapped onto and hence we present the metrics without probing into their relative importance. In the next section, we present an overview of the methods that are compared under the purview of road network partitioning.